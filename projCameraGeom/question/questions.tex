%%%%%%%%%%%%%%%%%%%%%%%%%%%%%%%%%%%%%%%%%%%%%%%%%%%%%%%%%%%%%%%%%%%%%
%
% VC36O Writeup Template
%
% This is a LaTeX document. LaTeX is a markup language for producing 
% documents. Your task is to fill out this
% document, then to compile this into a PDF document. 
% You will then upload this PDF to `Moodle'.
%
% 
% TO COMPILE:
% > pdflatex thisfile.tex
%
% For references to appear correctly instead of as '??', you must run 
% pdflatex twice.
%
% If you do not have LaTeX and need a LaTeX distribution:
% - Personal laptops (all common OS): www.latex-project.org/get/
%
% If you need help with LaTeX, please come to office hours. 
% Or, there is plenty of help online:
% https://en.wikibooks.org/wiki/LaTeX
%
% Good luck!
%
%%%%%%%%%%%%%%%%%%%%%%%%%%%%%%%%%%%%%%%%%%%%%%%%%%%%%%%%%%%%%%%%%%%%%%%%%%%%%%%%%%%%%%%%%%%%%%%%
%
% How to include two graphics on the same line:
% 
% \includegraphics[width=0.49\linewidth]{yourgraphic1.png}
% \includegraphics[width=0.49\linewidth]{yourgraphic2.png}
%
% How to include equations:
%
% \begin{equation}
% y = mx+c
% \end{equation}
% 
%%%%%%%%%%%%%%%%%%%%%%%%%%%%%%%%%%%%%%%%%%%%%%%%%%%%%%%%%%%%%%%%%%%%%%%%%%%%%%%%%%%%%%%%%%%%%%%%

\documentclass[11pt]{article}

\usepackage[english]{babel}
\usepackage[utf8]{inputenc}
\usepackage[colorlinks = true,
            linkcolor = blue,
            urlcolor  = blue]{hyperref}
\usepackage[a4paper,margin=1.5in]{geometry}
\usepackage{stackengine,graphicx}
\usepackage{fancyhdr}
\setlength{\headheight}{15pt}
\usepackage{microtype}
\usepackage{times}

% From https://ctan.org/pkg/matlab-prettifier
\usepackage[numbered,framed]{matlab-prettifier}

\frenchspacing
\setlength{\parindent}{0cm} % Default is 15pt.
\setlength{\parskip}{0.3cm plus1mm minus1mm}

\pagestyle{fancy}
\fancyhf{}
\lhead{Project 4 Questions}
\rhead{VC36O 2018/1}
\rfoot{\thepage}

\date{}

\title{\vspace{-1cm}Project 3 Questions}



\begin{document}
\maketitle
\vspace{-3cm}
\thispagestyle{fancy}

%\section*{Instructions}
%\begin{itemize}
%  \item 3 questions.
%  \item Write code where appropriate.
%  \item Feel free to include images or equations.
%  \item \textbf{Please use only the space provided and keep the page breaks.} Please do not make new pages, nor remove pages. The document is a template to help grading.
%  \item If you really need extra space, please use new pages at the end of the document and refer us to it in your answers.
%\end{itemize}

\section*{Questions}

\paragraph{Q1:} Describe the difference between the essential and fundamental matrices.

%%%%%%%%%%%%%%%%%%%%%%%%%%%%%%%%%%%
\paragraph{A1:} As matrizes Essencial e Fundamental descrevem a relação geométrica entre os pontos correspondentes de um par de câmeras estéreo. A única diferença entra as duas é que a primeira lida com câmeras calibradas, enquanto a segunda lida com câmeras não calibradas. A matriz Essencial contém cinco parâmetros (três para rotação e dois para a direção da tradução - a magnitude da tradução não pode ser recuperada devido à ambiguidade profundidade ou velocidade) e tem duas restrições: (1) seu determinante é zero e (2) seus dois valores singulares diferentes de zero são iguais. Já a matriz Fundamental contém sete parâmetros (dois para cada um dos epipolos e três para a homografia entre os dois feixes de linhas epipolares) e sua classificação é sempre dois \cite{Stanley}.


%%%%%%%%%%%%%%%%%%%%%%%%%%%%%%%%%%%

% Please leave the pagebreak
\pagebreak
\paragraph{Q2:} What does it mean when your epipolar lines: a) cross at more than one point, b) radiate out of a point on the image plane, or c) converge to a point outside of the image plane?

%%%%%%%%%%%%%%%%%%%%%%%%%%%%%%%%%%%
\paragraph{A2:} Your answer here.

Quando as linhas epipolares cruzar em mais de um ponto, acontece intersecções do plano epipolar com planos de imagem \cite{hartley2003multiple}.

Quando as linhas epipolares irradiam de um ponto no plano da imagem, o epipolo neste caso é denominado o foco de expansão e as mesmas linhas epipolares são sobrepostas em ambos os casos \cite{hartley2003multiple}.

Quando as linhas epipolares convergem para um ponto fora do plano da imagem, os epipolos são infinitos e linhas epipolares são paralelas \cite{hartley2003multiple}.


%%%%%%%%%%%%%%%%%%%%%%%%%%%%%%%%%%%

% Please leave the pagebreak
\pagebreak
\paragraph{Q3:} What is rectification, and why do we rectify image pairs?

%%%%%%%%%%%%%%%%%%%%%%%%%%%%%%%%%%%
\paragraph{A3:} A retificação de imagem é um processo de transformação usado para projetar imagens em um plano de imagem comum. Esse processo possui vários graus de liberdade e há muitas estratégias para transformar imagens no plano comum \cite{wiki}.

Dado um par de imagens estéreo, a retificação determina uma transformação de cada plano de imagem de modo que os pares de linhas epipolares conjugadas se tornem colineares e paralelas a um dos eixos da imagem. As imagens retificadas podem ser consideradas como adquiridas por uma nova sonda estéreo, obtida girando as câmeras originais ao redor do centro óptico. A importante vantagem da retificação é que as correspondências de computação, um problema de pesquisa 2-D em geral, são reduzidas a um problema de pesquisa 1-D, normalmente ao longo das linhas de varredura horizontal das imagens retificadas \cite{reti}.



%%%%%%%%%%%%%%%%%%%%%%%%%%%%%%%%%%%


% If you really need extra space, uncomment here and use extra pages after the last question.
% Please refer here in your original answer. Thanks!
%\pagebreak
%\paragraph{AX.X Continued:} Your answer continued here.


\bibliographystyle{abntex2-alf}
\bibliography{references}

\end{document}
